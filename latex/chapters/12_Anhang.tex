\chapter*{Anhang}
\addcontentsline{toc}{chapter}{Anhang}
\label{ch:anhang}
\vspace{-5mm}
\section*{Tabelle A1: Liste der Agenten}
\label{tabe1}

Zu sehen ist eine vollständige Liste aller fünfzehn verwendeter \acs{DRL} Varianten und der zwei Baselines. Die ersten vier Varianten werden näher untersucht.
Die Agenten sind entsprechend ihrer Performance in der Vorstudie sortiert. Deep curiosity q learning und verschiedene Modifikationen davon verwenden einen weiteren \acs{DRL} Algorithmus, der unter anderem in \parencite{10.5555/1671238} näher erläutert wird. Aufgrund der schlechteren Performance in der Vorstudie im Vergleich zu anderen Implementierungen wird auf diese Varianten daher nicht näher eingegangen. Außerdem erweist sich die Double Strategie (cf. \parencite{vanhas}) als nicht Vorteil für den algorithmischen Handel in diesem Setting.
\vspace{1mm}
\begin{table}[h!]
\label{fig:table1h}
\begin{adjustwidth}{-.8in}{0in}
  \begin{center}
  \hline
  \vspace{5}
	\begin{tabular}{l}
	duel deep recurrent q learning\\
	deep q learning\\
	deep deterministic policy gradient\\
	deep deterministic policy gradient mit Kontextdaten\\
	\vspace{-2mm}\\
	deep recurrent q learning\\
	double duel deep recurrent q learning\\
	deep recurrent deterministic policy gradient\\
	double deep recurrent q learning\\
	duel deep q learning\\
	duel deep recurrent deterministic policy gradient\\
	double duel deep q learning\\
	deep recurrent curiosity q learning\\
	duel deep curiosity q learning\\
	double deep q learning\\
	duel deep deterministic policy gradient\\
	deep curiosity q learning\\
	turtle trading\\
	moving average
	  \vspace{8}
    \end{tabular}
    \hline
    \vspace{10}
  \end{center}
  \end{adjustwidth}
\end{table}
\vspace{-3mm}
\section*{Tabelle A2: Datensatz}
\label{tabe2}
\addcontentsline{lot}{table}{A1~~~Vollständige Liste aller \acs{DRL} Varianten}

Der nachstehende Datensatz umfasst alle Wertpapiere für die Vorstudie (grau hinterlegt), die Hauptevaluation und des Krisenjahres (unterstrichen). Weiterhin sind Kontextdaten enthalten, die dem \acs{DDPGAK} und dem \acs{EDQLA} zusätzlich übergeben werden.
Die Wertpapiere werden zur besseren Übersicht in Wertpapierklassen und Börsen unterteilt. Absätze gruppieren die Wertpapiere hinsichtlich des Sektors, der Unternehmensgröße und der Region. Hinter den geschweiften Klammern steht der weitere Vertreter des Sektors, dessen Wertentwicklung beim Handel mit Wertpapieren dieses Sektors zusätzlich in den Kontextdaten enthalten sind.
\enlargethispage{2\baselineskip}
\vspace*{-4mm}%
\begin{table}[!b]
\label{tab:allassets}
\begin{adjustwidth}{-.8in}{-.5in}
\rule{19.4cm}{1}\vspace{-3mm}
\begin{minipage}[t]{.48\linewidth}
\centering
\footnotesize
\label{tab:first_table}
\begin{tabular}{ll}
\header{~~~Ticker}  &  \header{Unternehmen} \\   
\midrule

\textbf{Aktien (USA)}
\vspace*{1.6mm}\\
~~~\colorbox{grey!20}{ORCL}	&\colorbox{grey!20}{Oracle}\hspace{8em}\rdelim\}{16.4}{17.5mm}[\parbox{12.5mm}{\scriptsize{Broadcom\\(AVGO)}}]\\
~~~\colorbox{grey!20}{AAPL} 	&\colorbox{grey!20}{Apple}\\
~~~~ACN	& Accenture\\
~~~~MSFT	& Microsoft\\
~~~~IBM	& IBM\\
~~~~CSCO	& Cisco~~~~~~~~~~~~~~~~~~~~~~~~~~~~~~~~~~~~~~~~~~~~~~~\\
~~~~NVDA	& Nvidia\\
~~~~ADBE	& Adobe\\
~~~~HPQ	& HP\\
~~~~\underline{INTC}	& \underline{Intel}  
\vspace*{1.4mm}\\
~~~~TESS	& TESSCO Technologies\\
~~~~ASYS	& Amtech Systems\\
~~~~CTG	& Computer Task Group\\
~~~~BELFB	& Bel Fuse\\
~~~~AVNW	& Aviat Networks\\
~~~~LYTS	& LSI Industries
\vspace*{1.4mm}\\
~~~\colorbox{grey!20}{\underline{JPM}}	& \colorbox{grey!20}{\underline{JPMorgan Chase}}\hspace{1em}\rdelim\}{3}{17.5mm}[\parbox{12.5mm}{\scriptsize{Blackrock\\(BLK)}}]\\
~~~~BAC	& Bank of America\\
~~~\colorbox{grey!20}{\underline{V}} & \underline{Visa Inc.}
\vspace*{1.4mm}\\
~~~\colorbox{grey!20}{PFE}	& 	\colorbox{grey!20}{Pfizer}\hspace{7em}\rdelim\}{3}{17.5mm}[\parbox{12.5mm}{\scriptsize{AstraZeneca\\(AZN)}}]\\
~~~~MRK	& 	Merck \& Co.\\
~~~~JNJ		& Johnson \& Johnson
\vspace*{1.4mm}\\
~~~\colorbox{grey!20}{CAJ}	& 	\colorbox{grey!20}{Canon}\hspace{7.5em}\rdelim\}{9.5}{17.5mm}[\parbox{120mm}{\scriptsize{China Telecom\\(CHA)}}]\\
~~~~\underline{NICE}	& 	\underline{NICE}\\
~~~~TSM	& 	Taiwan Semiconductor\\
~~~~SNE	& 	Sony\\
~~~~UMC	& 	United Microelectronics\\
~~~~CHKP	& 	Check Point Software
\vspace*{1.4mm}\\
~~~~SILC	& 	Silicom\\
~~~~GILT	& 	Gilat Satellite Networks\\
~~~~TSEM	& 	Tower Semiconductor
\vspace*{1.4mm}\\
~~~~LFC	& 	China Life Insurance\hspace{1em}\rdelim\}{3}{17.5mm}[\parbox{120mm}{\scriptsize{Nomura Holdings\\(NMR)}}]\\
~~~~\underline{SMFG}	& 	\underline{Sumitomo Mitsui} \\
~~~~SHG	& 	Shinhan
\vspace*{1.4mm}\\
~~~~NOK	& 	Nokia\hspace{5em}\rdelim\}{6}{17.5mm}[\parbox{120mm}{\scriptsize{STMicroelectronics\\(STM)}}]\\
~~~~ASML	& 	ASML\\
~~~~ERIC	& 	LM Ericsson\\
~~~~SAP	& 	SAP\\
~~~~TEL	& 	TE Connectivity\\
~~~~LOGI	& 	Logitech
\vspace*{1.4mm}\\
~~~\colorbox{grey!20}{HSBC}	& 	\colorbox{grey!20}{HSBC}\hspace{5em}\rdelim\}{3}{17.5mm}[\parbox{120mm}{\scriptsize{Credit Suisse\\(CS)}}]\\
~~~~ING	& 	ING\\
~~~~BCS	& 	Barclays\\

\end{tabular}
\end{minipage}%
\begin{minipage}[t]{.5\linewidth}
\centering
\footnotesize
\label{tab:first_table}
\begin{tabular}{ll}
\header{~~~Ticker}  &  \header{Unternehmen} \\ 

\midrule

\textbf{Aktien (Asien)}
\vspace*{1.6mm}\\
~~~~0992.HK	& 	Lenovo Group\hspace{1em}\rdelim\}{3}{17.5mm}[\parbox{120mm}{\scriptsize{Tencent\\(0700.HK)}}]\\
~~~~3888.HK	& 	Kingsoft\\
~~~~0763.HK	& 	ZTE
\vspace*{1.4mm}\\
~~~~0939.HK	& 	China Construction Bank\hspace{0.4em}\rdelim\}{3}{17.5mm}[\parbox{120mm}{\scriptsize{China Merchants Bank\\(3968.HK)}}]\\
~~~~2318.HK	& 	Ping An Insurance\\
~~~~0998.HK	& 	China CITIC Bank
\vspace*{3.3mm}\\

\textbf{ETF's}
\vspace*{1.6mm}\\
~~~~\underline{DIA}	& 	\underline{Dow Jones}\hspace{1.7em}\rdelim\}{3}{17.5mm}[\parbox{120mm}{\scriptsize{NYSE Composite\\( $\hat{}$ NYA)}}]\\
~~~~SPY	& 	S\&P 500\\
~~~~QQQ		& 	NASDAQ
\vspace*{1.4mm}\\
~~~\colorbox{grey!20}{EWJ}	& 	\colorbox{grey!20}{Nikkei 225}\hspace{7em}\rdelim\}{3}{17mm}[\parbox{120mm}{\scriptsize{HANG SENG Index\\( $\hat{}$ HSI)}}]\\
~~~~EWT		& 	iShares MSCI Taiwan\\
~~~~\underline{EWY}	& 		\underline{iShares MSCI South Korea}
\vspace*{1.4mm}\\
~~~~\underline{EZU}	& 	\underline{FTSE 100}\hspace{4.5em}\rdelim\}{3}{17.5mm}[\parbox{120mm}{\scriptsize{IBEX\\(IBEX)}}]\\
~~~~CAC.PA	& 	Lyxor CAC 40\\
~~~~EXS1.DE	& 	iShares Core DAX
\vspace*{1.4mm}\\
~~~~EXXY.MI	& iShares Diversified Commodity\hspace{0.3em}\rdelim\}{2}{17.5mm}[\parbox{120mm}{\scriptsize{iShares S\&P GSCI\\(GSG)}}]\\
~~~~DBC		& 	Invesco DB Commodity Index~~~~~~~~~~~~~~~~~~~~~~~~~~
\vspace*{3.3mm}\\

\textbf{Waren}
\vspace*{1.6mm}\\
~~~\colorbox{grey!20}{GC=F} 	& 	\colorbox{grey!20}{Gold}\hspace{3em}\rdelim\}{4}{17.5mm}[\parbox{120mm}{\scriptsize{Aluminum\\(ALI=F)}}]\\
~~~~SI=F	& 	Silver\\
~~~~PL=F	& 	Platinum\\
~~~~HG=F	& 	Copper
\vspace*{1.4mm}\\
~~~\colorbox{grey!20}{\underline{CL=F}}	& 	\colorbox{grey!20}{\underline{Crude Oil}}\hspace{3em}\rdelim\}{4}{17.5mm}[\parbox{120mm}{\scriptsize{Brent Crude Oil\\(BZ=F)}}]\\
~~~~HO=F	& 	Heating Oil\\
~~~~NG=F	& 	Natural Gas\\
~~~~RB=F	& 	RBOB Gasoline
\vspace*{1.4mm}\\
~~~~HE=F	& 	Lean Hogs\hspace{3em}\rdelim\}{3}{17.5mm}[\parbox{120mm}{\scriptsize{Class III Milk\\(DC=F)}}]\\
~~~~LE=F	& 	Live Cattle\\
~~~~GF=F	& 	Feeder Cattle\\
~~~~\underline{DC=F}	& 	\underline{Class III Milk} $\longrightarrow$ ~\scriptsize{Live Cattle (LE=F)} 
\vspace*{1.4mm}\\
~~~~ZC=F	& 	Corn\hspace{6em}\rdelim\}{4}{17.5mm}[\parbox{120mm}{\scriptsize{Cocoa\\(CC=F)}}]\\
~~~~ZS=F	& 	Soybean\\
~~~~KC=F	& 	Coffee\\
~~~~KE=F	& 	KC HRW Wheat
\vspace*{3.3mm}\\

\textbf{Kontextdaten}
\vspace*{2mm}\\
~~~~$\hat{}$ SPY & S\&P 500 Trust\\
~~~~$\hat{}$ TNX & Treasury Yield\\
~~~~$\hat{}$ VIX & Volatility Index\\
~~~~GC=F & Gold Feb 21

\end{tabular}
\end{minipage}%
\vspace*{1mm}
\\
\rule{19.4cm}{1}
\end{adjustwidth}
\vspace*{3mm}\\
\begin{adjustwidth}{-.2in}{-15in}
\captionsetup{width=1.15\textwidth}
\end{adjustwidth}
\end{table}
\addcontentsline{lot}{table}{A2~~~Übersicht über alle verwendeten Wertpapiere}
