\PassOptionsToPackage{table,svgnames,dvipsnames}{xcolor}

\usepackage[utf8]{inputenc}
\usepackage[T1]{fontenc}
\usepackage[sc]{mathpazo}
\usepackage{lmodern}
\usepackage[ngerman]{babel}
\usepackage[autostyle]{csquotes}
\usepackage{multirow}
\usepackage[acronym,nomain,printonlyused, withpage]{glossaries}
\usepackage{setspace}
\usepackage{mathtools}
\usepackage{enumerate}
\RedeclareSectionCommand[beforeskip=0.7pt,
afterskip=1.1cm]{chapter}
\usepackage[hidelinks]{hyperref}

\newcounter{para}
\newcommand\mypara{\par\refstepcounter{para}\thepara\space}

\setcounter{table}{0}
\renewcommand{\thetable}{A\arabic{table}}

\usepackage{parskip}
\usepackage[font=scriptsize]{caption}
\RedeclareSectionCommand[
  beforeskip=.0\baselineskip,
  afterskip=-1em]{paragraph}

\usepackage[printonlyused, withpage]{acronym}
\usepackage{amsmath}
\usepackage[
  backend=biber,
  url=false,
  style=numeric,
  maxnames=4,
  minnames=3,
  maxbibnames=99,
  giveninits,
  uniquename=init]{biblatex}
\addbibresource{bibliography.bib}
\usepackage{graphicx}
\usepackage{scrhack} % necessary for listings package
\usepackage{listings}
\usepackage{lstautogobble}
\usepackage{booktabs} %Tabellen
\usepackage[final]{microtype} %verbessert Darstellung
\usepackage{caption}
\usepackage[hidelinks]{hyperref} % hidelinks removes colored boxes around references and links
\usepackage{fancyhdr}
\setlength{\headheight}{15.2pt}
\usepackage[justification=RaggedRight, singlelinecheck=false]{caption}
\usepackage{color}
\usepackage{tocloft}

\usepackage[hidelinks]{hyperref}
\renewcommand{\cftchapdotsep}{\cftdotsep} %adding dottet lines to table of contents
\newcommand{\E}{\mbox{I\negthinspace E}}


%Kopf- und Fußzeile
\pagestyle{headings}
\fancyhead{} 
\fancyfoot{}
\lhead[\leftmark]{}
\rhead[]{\leftmark}
\lfoot[\thepage]{}
\rfoot[]{\thepage}
%no lines header footer
\renewcommand{\headrulewidth}{0pt}
\renewcommand{\footrulewidth}{0pt}


\setkomafont{disposition}{\normalfont\bfseries} % use serif font for headings
\linespread{1.05} % adjust line spread for font

% Add table of contents to PDF bookmarks
\BeforeTOCHead[toc]{{\cleardoublepage\pdfbookmark[0]{\contentsname}{toc}}}
\setlength{\cftbeforetoctitleskip}{-1em}

\usepackage{tikz}
\usetikzlibrary{matrix, arrows.meta}
\usetikzlibrary{positioning,fit}
\usepackage{newfloat}
\DeclareFloatingEnvironment[fileext=dia,within=section]{diagram}

\newcommand{\DrawNeuronalNetwork}[2][]{
\xdef\Xmax{0}
\foreach \Schicht/\X/\Col/\Miss/\Lab/\Count/\Content [count=\Y] in {#2}
{\pgfmathsetmacro{\Xmax}{max(\X,\Xmax)}
 \xdef\Xmax{\Xmax}
 \xdef\Ymax{\Y}
}
\foreach \Schicht/\X/\Col/\Miss/\Lab/\Count/\Content [count=\Y] in {#2}
{\node[anchor=south] at ({2*\Y},{\Xmax/2+0.1}) {\Schicht};
 \foreach \m in {1,...,\X}
 {
  \ifnum\m=\Miss
   \node [neuron missing] (neuron-\Y-\m) at ({2*\Y},{\X/2-\m}) {};
  \else
   \node [circle,draw,minimum size=0.8cm] (neuron-\Y-\m) at 
  ({2*\Y},{\X/2-\m}) {\Content};
 \ifnum\Y=1
  \else
   \pgfmathtruncatemacro{\LastY}{\Y-1}
   \foreach \Z in {1,...,\LastX}
   {
    \ifnum\Z=\LastMiss
    \else
     \draw[->] (neuron-\LastY-\Z) -- (neuron-\Y-\m);
    \fi
    }
  \fi
 \fi
 \ifnum\Y=1
  \ifnum\m=\X
   \draw [overlay] (neuron-\Y-\m) -- (state);
  \else
   \ifnum\m=\Miss
   \else
    \draw [overlay] (neuron-\Y-\m) -- (state);
   \fi
  \fi
 \else
 \fi     
 }
 \xdef\LastMiss{\Miss}
 \xdef\LastX{\X}
}
}


%Code
\usepackage{algorithm}
\usepackage{algpseudocode}



%Listen
\usepackage{lineno}
\modulolinenumbers[5]
\usepackage{graphicx}
\usepackage{booktabs}
\usepackage{amssymb,amsmath,nccmath}
\usepackage{cclicenses}
\usepackage{makecell}
\usepackage{textcomp}
\usepackage{amsmath}
\usepackage{bigdelim} 
\usepackage{lipsum}
\usepackage{xcolor}

\usepackage{lscape,array}
\newcolumntype{C}[1]{>{\centering\arraybackslash}p{#1}} 
\usepackage[thin, , thinc]{esdiff}
\usepackage{subcaption}
\usepackage{caption}
\usepackage{framed}  
\usepackage[font=small,skip=0pt]{caption}

\newcommand{\header}[1] {
  \footnotesize{\textbf{#1}}}
\usepackage{geometry}
\usepackage[showframe=true]{geometry}
\usepackage{changepage}	



%not used:
%\usepackage{pgfplots}
%\usepackage{pgfplotstable}